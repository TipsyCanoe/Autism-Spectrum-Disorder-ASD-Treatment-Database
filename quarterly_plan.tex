\documentclass{article}
\usepackage[utf8]{inputenc}
\usepackage{geometry}
\geometry{a4paper, margin=1in}
\usepackage{graphicx}
\usepackage{tabularx}
\usepackage{booktabs}
\usepackage{hyperref}

\title{Quarter Project Plan}
\author{[Sendan Center Autism Resource Database]}
\date{Fall 2025}

\begin{document}

\maketitle

\section*{Instructions}
In this assignment, you will create a detailed plan for your final quarter of Senior Project.  
This plan must include:  
\begin{itemize}
    \item A clear summary of your project’s current state.  
    \item A roadmap describing your goals for this quarter and why they matter.  
    \item A list of specific deliverables to complete by the end of the quarter.  
    \item A customized rubric that will be used to assess your final deliverables.  
    \item Defined roles and responsibilities for each team member.  
    \item Documentation of your working spaces, communication tools, and meeting times.  
\end{itemize}

Your plan should be **concrete and measurable**—avoid vague goals like "improve code quality" and instead write "achieve 85\% test coverage on all modules."  

\section*{Project Current State}
This project has a frontend, backend database, integration, and a working search function. We currently have around 80\% code coverage regarding testing. Currently hosted by WWU at star.cs.wwu.edu. We have more to implement, primarily regarding database expansion, frontend completion (filling in non-critical gaps), and narrowing the search function for more accurate results.

\section*{Project Roadmap}
\textbf{Weeks 1-3}: We will be using this time to reorganize ourselves with the project, and meeting with our customer to determine remaining requirements. Through initial meetings, we can also determine what current deployment modules are unsatisfactory. We also hope to iron out some kinks regarding the frontend, and regain our footing with the Database to begin using it dynamically instead of showing only a certain number of results live. 
\\\\
\textbf{Weeks 4-6}: Majority of work done here. Fully complete frontend-backend connection, fleshed out frontend. Automation worked on here to get updates automated from Github to live server, instead of manual. Refine our summary generated by our local LLAMA model to make it more accurate. 
\\\\
\textbf{Weeks 7-10}: Ironing out kinks. Manual testing and quality control. Ideally we have real world testing to make sure that everything works correctly and how it should feel. Making sure server usage doesn't risk a crash under load, so that systems become dysfunctional. Certifying search function reports live, accurate results and our database is functional.

\section*{Project Deliverables}
% List out the specific things you will have completed by the end of the quarter.
\begin{itemize}
    \item Website deployed on WWU, accessible to all public.
    \item Database fully integrated with website, papers are relevant to searched topic and summary contains accurate info pertaining to statistics of research.
    \item Minimum of 85\% unit test coverage for backend modules and 70\% for frontend components.  
    \item Documentation package including: user manual, developer guide, deployment instructions.  
    \item Final presentation slides and demo video.  
\end{itemize}



\section*{Final Deliverables Rubric}
Your rubric should include at least 5--6 criteria. Adapt the criteria to your project’s scope.  

\begin{tabularx}{\textwidth}{|X|X|X|X|X|}
    \hline
    \textbf{Criterion} & \textbf{Excellent} & \textbf{Good} & \textbf{Needs Improvement} & \textbf{No Submission}\\
    \hline
    Accuracy & Search results pertain to the desired topic, summary is relevant and adds to the abstract & Search results pertain to the desired topic, summary is relevant but does not add to abstract & Search results pertain to the desired topic, summary is not relevant & Search results are not relevant to the desired topic, summary is not relevant\\
    \hline     
    Performance & Results are obtained within 5 seconds, continues functioning when updating database & Results are obtained within 10 seconds, continues functioning when updating database & Results are obtained over 10 seconds, ceases functioning when updating database & Searches fail to return anything. \\
    \hline
    Automation & Updates for database and general codebase is fully automatic & Updates for database and general codebase is mostly automatic (e.g. manual restart for errors or unexpected shutdowns) & Updates for database is automatic, general codebase is not & Updates for database and general codebase must be done manually \\
    \hline     
     Test Coverage & $\geq$ 85\% backend and $\geq$ 70\% frontend test coverage & 70--84\% backend and 50--69\% frontend test coverage & Less than 70\% backend or 50\% frontend test coverage & No testing evidence. \\
    \hline
    Documentation & Documentation fully understandable for customer, able to use guide to troubleshoot problems, know how to reach support if needed &  Documentation mostly understandable for customer, sometimes reaches support for regular maintenance (errors or changing update criteria) &  Documentation largely not useful for customer, regularly reaches out for support & Did not demonstrate this criterion. \\
    \hline
    % Add more rows as needed
    
\end{tabularx}



\section*{Team Roles}
% Define the roles for each team member, including responsibilities and expectations. Ensure that roles are aligned with the project's needs and each member's skills.


\begin{tabularx}{\textwidth}{|X|X|X|}
    \hline
    \textbf{Team Member} & \textbf{Strength} & \textbf{Role and Responsibilities} \\
    \hline
    Alex Lo & Automated Retrieval of Papers & Focus on refining AI with Logan this quarter \\
    \hline
    Cole Oliva & Frontend and Backend Expert & Focus on refining frontend, backend works correctly\\
    \hline
    Richard Jefferson & Database Expert & Focus on making the database up and running, with correct info parsed \\
    \hline
    Logan Calloway & AI Expert & Focus on refining AI, and giving relevant results to backend \\
    \hline
    % Add more rows as needed
\end{tabularx}

\section*{Working Spaces, Communication, and Meeting Times}
\begin{itemize}
    \item GitHub Repository: \url{https://github.com/TipsyCanoe/Autism-Spectrum-Disorder-ASD-Treatment-Database}
    \item Project Files: \url{https://drive.google.com/drive/folders/1H8u0dCdWFGXVbgMcCw3vaxQKRc88_Zty}

    \item Team Meeting Times: Thursdays 11am--12pm

\end{itemize}

\section*{Role of AI}
%Describe whether you plan on using AI in your senior project and why. If yes, list the tools you intend to use and describe how you will use each of them. Reflect on the benefits and risks of using (or not using) AI.
We have used generative AI to create boiler plate code (in regarding testing, typing out "name = main" methods, etc.), and have used it to brainstorm examples of existing systems similar to what we are doing now. These examples are manually reviewed, and always treated with caution to what info they provide. AI can be a large help in reducing time for research and typing out tedious code, but there is always a risk of it providing false information. And when using generated code, make sure that one knows what it is generating and what does it mean.

\section*{Client Acknowledgment}
I have reviewed and approved this project plan.  

\vspace{1em}
\noindent\textbf{Name:} \underline{\hspace{7cm}}  

\vspace{1em}
\noindent\textbf{Signature (written or digital):} \underline{\hspace{7cm}}  

\vspace{2em}
\noindent\textit{Instructions for Students:}  
This plan must be reviewed and approved by your client.  
Your client may either:  
\begin{itemize}
    \item Sign this document directly (if printed), or  
    \item Approve it digitally by replying to an email containing this document.  
\end{itemize}

\end{document}
